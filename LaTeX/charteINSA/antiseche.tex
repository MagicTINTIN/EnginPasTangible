
un texte \footnote{avec un bas de page}
\begin{figure}[H]
  \centering
  \includegraphics[width=0.7\textwidth]{graphics/organigramme.png}
  \caption{desc de l'image}
  \label{structure_capital}
\end{figure}
un autre texte

\section{du code}

\begin{minted}[linenos,frame=single]{js}
var canvas = document.getElementById("Canvas");
var ctx = canvas.getContext("2d");

SetCamera(0, 0, 0)
SetCameraAngle(5, 5, 5)
SetDisplay(300, 300, 300)


document.addEventListener("keydown",keyPush);
setInterval(mainloop,1000/60);

i=0
function mainloop() {
	i = i+1
    ctx.fillStyle="white";
    ctx.fillRect(0,0,canvas.width,canvas.height);
}
\end{minted}

\begin{code}
    \begin{minted}[linenos,frame=single]{js}
        //produit scalaire 
        function dotProduct(u, v) {
          return u[0] * v[0] + u[1] * v[1] + u[2] * v[2];
        }
        
        // produit vectoriel en dimension 3
        function crossProduct(u, v) {
          const x = u[1] * v[2] - u[2] * v[1];
          const y = u[2] * v[0] - u[0] * v[2];
          const z = u[0] * v[1] - u[1] * v[0];
          return [x, y, z];
        }
    \end{minted}
    \captionof{listing}{Exemple de fonctions que nous avons recodés}
    
    \label{code:crossproduct_and_dotproduct}
\end{code}