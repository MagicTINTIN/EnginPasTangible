\subsection{Code : Outils mathématiques}

\begin{minted}[linenos,frame=single]{js}
    //multiplication de 2 matrices
    function multMatrix(m1, m2) {
        var result = [];
        for (var i = 0; i < m1.length; i++) {
            result[i] = [];
            for (var j = 0; j < m2[0].length; j++) {
                var sum = 0;
                for (var k = 0; k < m1[0].length; k++) {
                    sum += m1[i][k] * m2[k][j];
                }
                result[i][j] = sum;
            }
        }
        return result;
    }
    
    function dotProduct(u, v) {    //produit scalaire 
      return u[0] * v[0] + u[1] * v[1] + u[2] * v[2];
    }
    
    function vectorLength(v) {     //Norme
      return Math.sqrt(v[0] * v[0] + v[1] * v[1] + v[2] * v[2]);
    }
    
    function crossProduct(u, v) {    // produit vectoriel en dimension 3
      const x = u[1] * v[2] - u[2] * v[1];
      const y = u[2] * v[0] - u[0] * v[2];
      const z = u[0] * v[1] - u[1] * v[0];
      return [x, y, z];
    }
    
    // retourne le vecteur normal au triangle
    function TriangleToNormal(som1,som2,som3) {
      const u = [
        som2[0] - som1[0],
        som2[1] - som1[1],
        som2[2] - som1[2],
      ];
      const v = [
        som3[6] - som1[0],
        som3[7] - som1[1],
        som3[8] - som1[2],
      ];
      const normal = crossProduct(u, v);
      return normal;
    }
    
    function Normalise(u){    //normalise un vecteur
      return [u[0]/vectorLength(u), u[1]/vectorLength(u), u[2]/vectorLength(u)]
    }
\end{minted}

\clearpage

\subsection{Code : Projection}
\begin{minted}[linenos,frame=single]{js}

//a est le point à transformer
function TransformedPoint(ax, ay, az) {
	cx = Math.cos(camera_angle.x)
	cy = Math.cos(camera_angle.y)
	cz = Math.cos(camera_angle.z)
	sx = Math.sin(camera_angle.x)
	sy = Math.sin(camera_angle.y)
	sz = Math.sin(camera_angle.z)
	mat1 = [
		[1, 0, 0],
		[0, cx, sx],
		[0, (-1)*sx, cx]
	]
	mat3 = [
		[cz, sz, 0],
		[(-1)*sz, cz, 0],
		[0, 0, 1]
	]
	vect = [[ax-camera.x, 0, 0],
		[ay-camera.y, 0, 0],
		[az-camera.z, 0, 0]
	]

	d = multMatrix(multMatrix(mat1, mat3), vect)
	
	vectd = [d[0][0], d[1][0], d[2][0]] 

	return vectd
}

function JustProjectPoint(dx, dy, dz) {
	ex = display.x
	ey = display.y
	ez = display.z
	return [ez*dx/dz+ex, ez*dy/dz+ey]
}

function Project(x, y, z) {
	tp = TransformedPoint(x, y, z)
	visible_point = true
	if(tp[2]<0) {  //on affiche la face que si elle est devant nous
		visible_point = false
	}
	projection = JustProjectPoint(tp[0], tp[1], tp[2])
	return [projection[0], projection[1], visible_point]
}
\end{minted}

\clearpage

\subsection{Code : Gestion des faces}

\begin{minted}[linenos,frame=single]{js}
    //permet d'afficher une face à l'écran
    function DrawTriangle(co1, co2, co3, color) {
    	ctx.fillStyle = color//"transparent"
    	ctx.stokeStyle = "black"
            ctx.lineWidth   = 1;
    	point1 = Project(co1[0], co1[1], co1[2])
    	point2 = Project(co2[0], co2[1], co2[2])
    	point3 = Project(co3[0], co3[1], co3[2])
    
    	if(point1[2] && point2[2] && point3[2]) {
    		ctx.beginPath();
    		ctx.moveTo(point1[0], point1[1]);
    		ctx.lineTo(point2[0], point2[1]);
    		ctx.lineTo(point3[0], point3[1]);
    		ctx.lineTo(point1[0], point1[1]);
    		ctx.stroke()  // dessine le contour
    		ctx.fill()    //rempli la face avec la couleur souhaité
    		ctx.closePath()
    	}
    }
\end{minted}
\begin{minted}[linenos,frame=single]{js}
    //on ajoute une face à la liste des faces à afficher
    function AddTriangle(co1, co2, co3, color) {
    	triangle_list.push([co1, co2, co3, color, 0])
    }

    //On rend toutes les faces (on affiche toutes les faces)
    function DrawAllTriangle() {
        triangle_list = faceOrder(triangle_list)
        while((face = triangle_list.shift()) !== undefined) {
            //On affiche la face qu'on vient de retirer
            DrawTriangle(face[0],face[1],face[2],face[3])
        }
    }
\end{minted}

\clearpage
\subsection{Code : Gestion des variables}
\begin{minted}[linenos,frame=single]{js}
    var fov = 400
    
    var triangle_list = []
    
    var camera = {
        x: 0,
        y: 0,
        z: 0
    }
    
    var camera_angle = {
        x: 0, //phi sphérique (angle bas-haut)
        y: 0, // angle phi dans les angles d'Euler => laisser constant à 0
        z: 0  //theta sphérique (angle droite gauche)
    }
    
    var display = {
        x: 0,
        y: 0,
        z: 0
    }
    
    function SetCamera(x, y, z) {
        camera.x = x
        camera.y = y
        camera.z = z
    }
    
    function SetCameraAngle(x, y, z) {
        camera_angle.x = x
        camera_angle.y = y
        camera_angle.z = z
    }

    // Fonction qui permet de calculer l'écran (qui est lié
    // à la caméra et l'angle de la caméra).
    // Cette fonction est appelé à chaque itération
    function NewDisplay() {
        display.x = fov*Math.cos(camera_angle.x)*Math.cos(camera_angle.y)
        display.y = fov*Math.cos(camera_angle.x)*Math.sin(camera_angle.y)
        display.z = fov*Math.sin(camera_angle.x)
    }
    
\end{minted}


\clearpage
\subsection{Code : Gestion des "inputs" (souris et clavier)}
\begin{minted}[linenos,frame=single]{js}
    // Variables pour stocker les coordonnées précédentes de la souris
    var previousX = null;
    var previousY = null;
    
    var mouseeventon = false;
    
    document.addEventListener("keydown",keyPush);
    
    document.addEventListener("mousemove", handleMouseMove);
    document.addEventListener("wheel", handleMouseZoom);
    
    function handleMouseMove(event) {
        if(mouseeventon) {
            // Vérifier si les coordonnées précédentes existent
            if (previousX !== null && previousY !== null) {
                // Calculer le delta horizontal et vertical
                var deltaX = event.clientX - previousX;
                var deltaY = event.clientY - previousY;
        
                // Utiliser les valeurs de deltaX et deltaY à des fins quelconques
                camera_angle.x = camera_angle.x - deltaY*0.002
                camera_angle.z = camera_angle.z + deltaX*0.002
            }
        }
        
        // Mettre à jour les coordonnées précédentes avec les coordonnées actuelles
        previousX = event.clientX;
        previousY = event.clientY;
    }
    
    function handleMouseZoom(event) {
        if(mouseeventon) {
            // Vérifier si le déplacement de la souris est un zoom
            if (event.deltaY < 0) {
                // Zoom in (approche)
                fov = fov - event.deltaY*0.1;
            } else {
                // Zoom out (éloignement)
                fov = fov - event.deltaY*0.1;
            }
        }
    }
    
    function keyPush(evt) {
        switch(evt.keyCode) {
            case 40: // bas
            	camera_angle.x = camera_angle.x - 0.05
                break;
            case 38: // haut
            	camera_angle.x = camera_angle.x + 0.05
                break;
            case 39: // droite
            	camera_angle.z = camera_angle.z + 0.05
                break;
            case 37: // gauche
            	camera_angle.z = camera_angle.z - 0.05
                break;
    
            case 81 : // q
            	camera.x = camera.x - 1
                break;
            case 68: // d
            	camera.x = camera.x + 1
                break;
            case 69: // e
            	camera.y = camera.y - 1
                break;
            case 65: // a
            	camera.y = camera.y + 1
                break;
            case 83: // s
            	camera.z = camera.z - 1
                break;
            case 90: // z
            	camera.z = camera.z + 1
                break;
        }
    }
\end{minted}


\clearpage
\subsection{Code : Parsing et Chargement d'une map}
\begin{minted}[linenos,frame=single]{js}
    function strToMatrix(str) {
        str = str.replace(/[^0-9.\s\r\nvf]/g, "");
        matrix = str.split("\n").map(line => line.split(" "))
        v = []  //liste des coordonnées des sommets
        f = []  //liste des faces contenant les identifiants des sommets

        //les 6 première lignes sont des métadonnées
        for(let i=5; i<matrix.length; i++) {
            if(matrix[i][0] == 'f' || matrix[i][0] == 'v') {
                matrix[i][3] = matrix[i][3]//.slice(0, -1)
                if(matrix[i][0] == 'v') {
                    v.push(matrix[i])
                } else {
                    f.push(matrix[i])
                }
            }
        }

        //ici on convertie le tableau de string en tableau de flottant
        var v = v.map(function(row) {
          return row.map(function(value) {
            return parseFloat(value, 10);
          });
        });

        //ici on convertie le tableau de string en tableau d'entier
        var f = f.map(function(row) {
          return row.map(function(value) {
            return parseInt(value, 10);
          });
        });

        //symétrie car sinon la projection inverse certaines mesures
        v.forEach(function(face){
          face[1] = (-1)*face[1]
          face[3] = (-1)*face[3]
        });
    
        return [v, f]
    }
    
    document.getElementById('inputFile').addEventListener('change', function() {
        var file = new FileReader();
        file.onload = () => {
            //usermap est la variable qui stocke les faces importés
            usermap = strToMatrix(file.result) 
          console.log(usermap)
        }
        file.readAsText(this.files[0])
    });
\end{minted}

\clearpage
\subsection{Code : Main Code}
\begin{minted}[linenos,frame=single]{js}
var canvas = document.getElementById("Canvas");
var ctx = canvas.getContext("2d");

SetCamera(-27, -18, -24)
SetCameraAngle(1.594, 0, 1.4879999999999)

setInterval(mainloop);  //boucle infini de la fonction mainloop

//fonction appelé à chaque itération (à chaque rendu d'une image)
function mainloop() {
    NewDisplay()
    ReloadInformation()
    ctx.fillStyle="white";
    ctx.fillRect(0,0,canvas.width,canvas.height); //on efface tout

    //si la map a été importé, la liste des faces est dans usermap
    //on ajoute alors toutes les faces via la fonction AddTriangle
    if(typeof usermap != "undefined") {
        for(let i = 0; i < usermap[1].length; i++) {
            offsetX = 0
            offsetY = 0
            offsetZ = 0
            AddTriangle(
                [offsetX+usermap[0][usermap[1][i][1]-1][1],
                offsetY+usermap[0][usermap[1][i][1]-1][2],
                offsetZ+usermap[0][usermap[1][i][1]-1][3]
                ],
                [offsetX+usermap[0][usermap[1][i][2]-1][1],
                offsetY+usermap[0][usermap[1][i][2]-1][2],
                offsetZ+usermap[0][usermap[1][i][2]-1][3]
                ],
                [offsetX+usermap[0][usermap[1][i][3]-1][1],
                offsetY+usermap[0][usermap[1][i][3]-1][2],
                offsetZ+usermap[0][usermap[1][i][3]-1][3]
                ],
                "green"
            )
        }
    }
    
    DrawAllTriangle() // on Draw toutes les faces qu'on a ajouté à la liste

    //ici on bloque la caméra si on regarde trop haut
    if (camera_angle.x > 3.14) {
        camera_angle.x = 3.14
    }
    //ici on bloque la caméra si on regarde trop bas
    if (camera_angle.x<0) {
        camera_angle.x=0
    }
}
\end{minted}