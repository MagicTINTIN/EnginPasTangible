\subsection{Scène simple}
%\addcontentsline{toc}{section}{Scène simple}
Code à mettre dans un fichier, nommé \emph{example.fs} par exemple
\begin{lstlisting}[language=GLSL]
#version 330 core
in vec2 FragCoord;
in float Time;
in vec2 MousePos;
in vec3 CamPos;
in vec3 CamDir;

out vec4 FragColor;
//uniform sampler2D generalTexture;

float SDF_Box_Frame( vec3 p, vec3 b, float e )
{
       p = abs(p  )-b;
  vec3 q = abs(p+e)-e;
  return min(min(
      length(max(vec3(p.x,q.y,q.z),0.0))+min(max(p.x,max(q.y,q.z)),0.0),
      length(max(vec3(q.x,p.y,q.z),0.0))+min(max(q.x,max(p.y,q.z)),0.0)),
      length(max(vec3(q.x,q.y,p.z),0.0))+min(max(q.x,max(q.y,p.z)),0.0));
}

float SDF_Circle(vec3 p,float r){
    return length(p)-r;
}

float SDF_Global(vec3 p){
    return min(
        SDF_Box_Frame(p,vec3(.5,.5,.5),0.1),
        SDF_Circle(mod(p+vec3(.5),vec3(1.,1.,1.))-vec3(.5),.15));
}

vec4 Get_Impact(vec3 origin,vec3 dir){//must have length(dir)==1 
    vec3 pos=origin;
    float dist;
    for(int i=0;i<30;i++){
        dist=SDF_Global(pos);
        pos+=dist*dir;
        if(dist<=.01) return vec4(pos,1.);
        if(dist>=20.0) return vec4(pos,-1.);
    }
    return vec4(pos,-1.);
}

vec3 grad(vec3 p){
    vec2 epsilon = vec2(.01,0.);
    return normalize(vec3(SDF_Global(p+epsilon.xyy)-SDF_Global(p-epsilon.xyy),
    SDF_Global(p+epsilon.yxy)-SDF_Global(p-epsilon.yxy),
    SDF_Global(p+epsilon.yyx)-SDF_Global(p-epsilon.yyx)));
}

vec3 Get_Color(vec3 origin,vec3 dir){
    vec4 impact = Get_Impact(origin,dir);
    if(impact.w<0.) return vec3(.5,.7,1.);
    vec3 normale=grad(impact.xyz);
    return normale;
}

void main()
{
    vec3 lookingAt = vec3(0.,0.,0.);
    vec3 posCam    = vec3(3.*sin(Time*.5),0.,3.*cos(Time*.5));
    
    vec3 ez = normalize(lookingAt - posCam); //base orthonormee
    vec3 ex = normalize(cross(ez,vec3(0.,1.,0.)));
    vec3 ey = cross(ex,ez);
    
    vec3 dir = normalize(FragCoord.x * ex + FragCoord.y*ey + 1.*ez);
    
  FragColor=vec4(Get_Color(posCam,dir),1.);
}
\end{lstlisting}