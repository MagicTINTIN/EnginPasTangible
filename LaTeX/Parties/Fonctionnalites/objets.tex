\subsection{Volumes élémentaires}
Dans cette partie, on s'interresse aux fonctions Signed Distance Function (SDF) de differentes formes géométriques. On remarque que toutes ces fonctions supposent que l'objet géometrique est centré en $O$. Cela simplifie beaucoup les fontions et il est très facile de leur appliquer une translation ou autre transformation, comme indiqué dans \ref{subsec:Transformations} \nameref{subsec:Transformations}.
\subsubsection{Sphere}
La sphere a la plus simple des SDF. En effet, la distance entre un point $P$ et une sphere de rayon $r$ est donnée par 
\begin{align*}
distance=length(P)-r
\end{align*}
Avec $length(x)=\|x\|_2$.
\\\textbf{Remarque:} On rappelle que le volume est centré en O.
\subsubsection{Plan}
La distance entre un point $P$ et un plan de normale $N$ (normalisée) et de hauteur $h$ est donnée par 
\begin{alignat*}{1}
    distance=\left\langle P,N \right\rangle - h
\end{alignat*}
\subsubsection{Et bien d'autres\ldots}
Il existe de nombreuses autres fonctions SDF. Cependant, elles deviennent très vite complexes à comprendre.\\
Voici par exemple la fonction SDF pour un pavé avec $taille=(largeur, hauteur, profondeur) \in \mathbb{R}^3$
Cette fontion utilise le fait que la fonction |.| sur un vecteur équivaut à une symetrie par rapport aux plans xOy, xOz et yOz. Or un pavé centré en O possede ces trois plans de symetries. Ainsi, la distance d'un point $P$ à ce pavé est donnée par le code suivant:

\begin{lstlisting}[language=GLSL]
float SDF_Box(vec3 p, vec3 taille) {
	vec3 q=abs(p)-taille;
	return length(max(q,0.0))+ min(max(q.x,max(q.y,q.z)),0.0);
}
\end{lstlisting}