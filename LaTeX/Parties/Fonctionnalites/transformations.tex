\subsection{Transformations}
Pour toutes les transformations sur les points des volumes, il suffit d'appliquer la transformation inverse sur la position de ce point.\\
Par exemple, si l'on souhaite translater un point vers les $x$ croissants, il faut réduire d'autant la position en $x$.

\subsubsection{Rotation autour des axes XYZ}
\begin{align*}
    R(x,y,z)
    =& R_{ex}(x)R_{ey}(y)R_{ez}(z) \\
    = &
    \begin{bmatrix}
    cos(x) & -sin(x) & 0 \\
    sin(x) & cos(x) & 0 \\
    0 & 0 & 1 
    \end{bmatrix}
    \begin{bmatrix}
    cos(y) & 0 & sin(y) \\
    0 & 1 & 0 \\
    -sin(y) & 0 & cos(y) 
    \end{bmatrix}
    \begin{bmatrix}
    1 & 0 & 0 \\
    0 & cos(z) & -sin(z) \\
    0 & sin(z) & cos(z) 
    \end{bmatrix}\\
    = & \begin{bmatrix}
    cos(y)cos(z) & sin(x)sin(y)cos(z) & cos(x)sin(y)cos(z)+sin(x)sin(z) \\
    cos(y)sin(z) & sin(x)sin(y)sin(z)+cos(x)cos(z) & cos(x)sin(y)sin(z)-sin(x)cos(z) \\
    -sin(y) & sin(x)cos(y) & cos(x)cos(y)
    \end{bmatrix} \\
\end{align*}

\subsubsection{Duplication des volumes}

