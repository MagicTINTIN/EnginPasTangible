\newpage
\subsection{Calcul de la couleur à afficher}
\subsubsection{Calcul de normale}
Pour les calculs de couleurs, on va avoir besoin de la normale $\Vec{N}$ de la surface de l'objet "rencontré" par le rayon.
\begin{align*}
    \Vec{N}=normalize(\Vec{\nabla}SDF\_Scene(P))
\end{align*}
avec $normalize(\cdot )=\frac{\cdot }{\|\cdot \|}$\\
\\
\textbf{Remarque} : on utilise le fait que $SDF\_Scene$ est $\mathcal{C}^1(\mathbb{R}^3,\mathbb{R})$ sauf en certains points particuliers, comme par exemple à l'interieur d'un plan sans volume. Cependant, ces points critiques sont négligeables par rapport à la surface totale de l'objet. On évitera d'utiliser certaines figures, au profit d'autres ayant moins de discontinuités. Par exemple, on préférera utiliser un pavé très fin plutôt qu'un plan ayant deux dimensions.
\subsubsection{Calcul de l'éclairage}
On pose $LightPos \in \mathbb{R}^3$ la position d'une source de lumière et $LightColor$ sa couleur.
\\La contribution de cette source de lumière à l'éclairement d'un objet blanc de normale $\Vec{N}$ au point $\mathbf{P}$ est donnée par ce calcul : 
\begin{align*}
    Contribution=max(\left\langle \Vec{N},\ normalize(Lightpos-P) \right\rangle,\ 0) \times LightColor
\end{align*}