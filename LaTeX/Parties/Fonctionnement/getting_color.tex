\subsection{Calcul de la couleur à afficher}
\subsubsection{Calcul de normale}
Pour les calculs de couleurs, on va avoir besoin de la normale de la surface de l'objet "rencontré" par le rayon.
\begin{align*}
\Vec{N}=Normalize(\Vec{\nabla}SDF\_Scene(P))
\end{align*}
avec $Normalize(u)=\frac{u}{\|u\|}$\\
\\
\textbf{Remarque} : on utilise le fait que $SDF\_Scene$ est $\mathcal{C}^1(\mathbb{R}^3,\mathbb{R})$ sauf en certains points particuliers, comme par exemple à l'interieur d'un plan sans volume. On evitera donc d'utiliser certaines figures, au profit d'autres ayant des propriétés plus satisfaisantes.
\subsubsection{Calcul de l'éclairage}
On pose $SunDir \in \mathbb{R}^3$ la direction \textbf{normalisée} du soleil.