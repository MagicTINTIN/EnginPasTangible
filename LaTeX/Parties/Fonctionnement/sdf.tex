\subsection{Prérequis : Signed Distance Function}

Le Ray Marching repose sur une fonction appelée SDF : Signed Distance Function.
\\
 Cette fonction doit retourner pour tout point M(t) la distance entre ce point et l'objet le plus proche. Cette distance peut et doit être négative si le point est à l'interieur d'une figure. L'ajout du temps permet de créer des objets dont la position, la forme, etc. dépendent du temps.
\\
Ainsi, nous avons :
\begin{flalign*}
SDF : \mathbb{R}^4 &\rightarrow \mathbb{R}\\
    (x,y,z,t) &\xmapsto{} SDF(x,y,z,t)\\
\end{flalign*}

%%%%%%%%%%%%%%%%%%%%%%%%%%%%%%%%%%%%%%%%%%%%%%%%%%%%%%%%%%%%% il faut déplacer + changer ça vers la partie Fonctionnement.    
Ainsi, voici la SDF pour une scene 3D ne contenant qu'une sphere de centre o(t):
\begin{flalign*}
    SDF\_Sphere : \mathbb{R}^4 &\rightarrow \mathbb{R}\\
    P(x,y,z,t) &\xmapsto{} length(P-o(t)) - r
\end{flalign*} Avec r le rayon de la sphere et $length(\cdot ) = \| \cdot  \|$