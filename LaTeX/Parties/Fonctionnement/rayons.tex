\newpage
\subsection{Création des rayons projetés}
\subsubsection{Création d'une base orthonormée pour la caméra}

On suppose que l'on connait $tilt$ et $pan$ qui sont les angles de la caméra donnés par le curseur. 
\begin{align*}
    e_z =&  (cos(tilt)\times sin(pan) ,\ sin(tilt) ,\ cos(tilt)\times cos(pan))\\
    e_x =&  normalize(e_z \wedge (0,1,0) )\\
    e_y =& e_x \wedge e_z\\
\end{align*}

Avec $normalize(\cdot ) = \frac{\cdot }{\| \cdot  \|}$

\textbf{Remarque} : Cette base de vecteurs est orthonormée.
\subsubsection{Création des vecteurs directeur pour chaque pixel}
Avec $x$ et $y$ les coordonées du pixel et FOV le champ de vision $(x,y,FOV\in \mathbb{R})$: 
\begin{align*}
    Direction &= normalize(x\times e_x + y\times e_y + FOV\times e_z)\\
\end{align*}

On obtient ainsi un vecteur directeur pour chaque pixel de l'écran.
\\INSERER VECTEURS BLENDER